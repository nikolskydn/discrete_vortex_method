\documentclass[a4paper]{report}
\mag 1440
% \mag1440 увеличивает в 1.45 раз
\hoffset=6.9mm \voffset=5.0mm
\topmargin-11.6mm \oddsidemargin-1.4mm
\textwidth110mm \textheight166mm
\headsep6.9mm


\usepackage[utf8]{inputenc}
\usepackage[russian]{babel}
\usepackage{euscript}
\usepackage{amsmath}
\usepackage{amsthm}
\usepackage{amssymb}
\usepackage{graphicx}
\usepackage{citehack}
\usepackage{caption2}
\usepackage[colorlinks,urlcolor=blue,linktocpage=true,unicode]{hyperref}
\usepackage{algorithmic}
\usepackage{algorithm}
%\usepackage{russcorr}

\pagestyle{headings}

\makeatletter
% номер страницы в центре
\renewcommand{\@oddhead}{\hfil --- \thepage \ --- \hfil}
\makeatother

% сброс счетчика после каждой секции
\makeatletter \@addtoreset{equation}{section}


% номер формулы 
\renewcommand{\theequation}{\thesection.\arabic{equation}}

% счётчик для ссылок на рисунки
\newcounter{pict}[section]
\setcounter{pict}{1}

% команда для сброса счётчика рисунков
\newcommand{\pc}{\refstepcounter{pict}{}}

% номер рисунка в виде секция.номер
\renewcommand{\thepict}{\thesection.\arabic{pict}}

% заменяем для рисунков ':' после номера рисунка на '.'
\renewcommand{\captionlabeldelim}{.}

\newtheorem{theorem}{Теорема}[section]
\newtheorem{lemma}{Лемма}[section]


\numberwithin{algorithm}{section}
\floatname{algorithm}{Алгоритм}

% запрет переноса после символов бинарного отношения
\relpenalty=10000
% грубый способ избавления от длинных строк
\sloppy

\usepackage{listings}
\lstloadlanguages{C}

\lstset{extendedchars=true,
%        commentstyle=\it,.
%        stringstyle=\bf,.
        language=bash,
        belowcaptionskip=5pt}

\usepackage{euscript}

\graphicspath{{figure/}}

\begin{document}

\section{Общая постановка задачи}

Эволюция границы раздела разноплотных и разновязких жидкостей $L_t$ в однородных пористых средах 
может быть описана системой интегральных и дифференциальных уравнений~\cite{ndn2011,ndn2013}:
\begin{equation}
\label{maineq}
\begin{gathered}
g_t(\vec{r}_M,t)-2\lambda G[g_t,L_t] (\vec{r}_M,t)=2\lambda\varphi_(\vec{r}_M,t)+2\alpha\Pi(\vec{r}_M), 
\quad \vec{r}_M \in L_t,\\
\frac{\partial \vec{r}_M}{\partial t} = \vec{W}_0(\vec{r}_M,t) + \vec{V}_{2}[g_t,L_t] (\vec{r}_M,t), 
\quad \vec{r}_M \in L_t, \quad t>0,\\
r_{M} = r_0 \quad \text{при} \quad t=0.
\end{gathered}
\end{equation}
Здесь 
$G[g_t,L_t]$ и $V_{2\varepsilon}[g_t,L_t]$~--- операторы потенциала и скорости двойного слоя,
рапределенного по границе $L_t$ с плотностью $g_t$; 
$\varphi_0$ и $W_0$~--- потенциал и скорость невозмущенного течения;
параметры $\lambda=\frac{\mu_2-\mu_1}{\mu_2+\mu_1}$ и $\alpha=\frac{\rho_1-\rho_2}{\rho_1+\rho_2}$ 
характеризуют различия в вязкостях $\mu_1$,$\mu_2$ и в плотностях $\rho_1$, $\rho_2$ 
совместно перемещающихся жидкостей;
$\Pi$~--- потенциал массовых сил; $t$~--- время; $\vec{r}_M$~--- радиус-вектор точки $M$.

Более подробную информацию см. в~\cite{ndn2011,ndn2013}.

Для численного решения систем \eqref{maineq} предлагаетя библиотека \verb|libdeu2d.so| (см. \verb|lib|), 
позволяющая  вычислять основные операторы $G$ и $V$, а так же пример ее использования (см. \verb|modelling|).

Численное решение задачи можно разбить на 3 этапа:
\begin{enumerate}
\item заполнение матрицы ;
\item решение СЛАУ;
\item вычисление смещения границы.
\end{enumerate} 

Требуется, использую стандарт OpenMP~\cite{omp} распараллелить работу программ.

\section{Комплекс параллельных программ для исследования плоскопараллельных процессов эволюции границы раздела разновязких жидкостей на универсальных многоядерных процессорах}

Упростить основную систему уравнений \eqref{maineq} --- взять $\alpha=0$. 
При распараллеливании этапа 2 <<Решение СЛАУ>> можно ограничиться простым методом Гаусса, полагая  $| \lambda | < 0.5$.

\section{Параллельное программное обеспечение для математического моделирования плоскопараллельных процессов эфолюции границы раздела разноплотных жидкостей на универсальных многоядерных процессорах}

Упростить основную систему уравнений \eqref{maineq} --- взять $\lambda=0$. При этом этап 2 
<<Решение СЛАУ>>  не требуется.


\section{Порядок использования}

Сборка проекта в каталоге \verb|modelling|:
\begin{enumerate}
    \item укажите путь к библиотекe --- \verb*|export LD_LIBRARY_PATH="$LD_LIBRARY_PATH:../lib"|;
    \item запустите сборку --- \verb|make|.
\end{enumerate}


Запуск и исследования:
\begin{enumerate}
    \item укажите число потоков --- \verb|export OMP_NUM_THREADS=1|;
    \item запустите программу  --- \verb|./one_boundary_diff_dens_and_visc b0|.
\end{enumerate}



\begin{thebibliography}{99}
    \label{sec:literature}
    \addcontentsline{toc}{section}{Список литературы}
    \bibitem{ndn2011}
        \url{http://www.ict.nsc.ru/jct/content/t15n1/Nikolsky_n.pdf}
    \bibitem{ndn2013}
        \url{https://github.com/nikolskydn/discrete_vortex_method_in_2D/blob/master/docs/nikolskydn-meth-mm.pdf}
    \bibitem{omp} \url{http://www.openmp.org/}
\end{thebibliography}

\end{document}

